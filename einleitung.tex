\section{Einleitung}
\label{sec:einleitung}

\subsection{Begründung der Problemstellung}

Online--Shopping ist inzwischen in der Gesellschaft angekommen und hat im Endkundenbereich den klassischen Katalog--Versandhandel weitgehend abgelöst. Dennoch sind Print--Broschüren und Kataloge weiterhin ein wichtiges Informartionsmedium für Kunden von stationären Handelsgeschäften. Als Katalog--Versandhändler mit Ladengeschäften und Internetshop ist es wichtig die unterschiedlichen Vertriebswege so zu kombinieren, dass dem Kunden ein möglichst passgenaues Einkaufserlebnis zu bieten.

\subsection{Ziele dieser Arbeit}

\textbf{Ziel dieser Arbeit ist es, Möglichkeiten der Verzahnung von Printkatalogen, Ladengeschäften und Onlineshop aufzuzeigen.}

Zunächst werden im Kapitel~\myref{sec:grundlagen} die relevanten Begriffe definiert sowie die Verknüpfungsrichtungen beschrieben und --möglichkeiten genannt.

Im Kapitel~\myref{sec:hauptteil} werden dann die informationstechnisch relevanten Verknüpfungsmethoden „Internetadresse“, „Verfügbarkeitsanzeige“ und „Info--Terminals“ genauer betrachtet sowie ein Handlungsempfehlungen für deren Einsatz bei „Manufactum“, einem Versandhändler für hochwertige Produkte, der neben Katalogen auch mehrere Ladengeschäfte sowie einen Onlineshop betreibt, gegeben.

\subsection{Abgrenzung}

In dieser Arbeit liegt das Augenmerk auf den Möglichkeiten, die unterschiedlichen Vertriebskanäle miteinander zu verknüpfen. Hierbei wird davon ausgegangen, dass diese Verknüpfungen positive auswirkungen auf das Kundenverhalten bzw. das Betriebsergebnis haben. 

Des weiteren werden speziell diejenigen Verknüpfungen untersucht, bei denen die Datenverarbeitung einen relevanten Beitrag leistet.

