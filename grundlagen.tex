\section{Grundlagen}
\label{sec:grundlagen}

\subsection{Akteur auf der Einzelhandelsebene}

\todo{Siehe \cite{wirtz} Kapitel 3.2.2, S.33, Abbiltung 3--6}

\subsection{Retailing}

Engl. für EInzelhandel, siehe \url{http://www.oxforddictionaries.com/definition/english/retail}, table 1

bzw. \url{http://link.springer.com/chapter/10.1007/978-3-8349-9160-7_24}

\todo{Marteking oder Retailing? Siehe \cite{wirtz}, Kapitel 2.2.4, S. 20f, Tabelle 2--7

Nur Distribution, nicht alle 4P.
}



\subsection{Definition Mehrkanalsystem}

Click, Bricks and Sheets, siehe \url{http://www.arraydev.com/commerce/jibc/2005-02/schramm-klein.htm}

\subsection{Definition Multichannel Marketing}

In dieser Arbeit wird der Begriff Multichannel Marketing gemäß der Definition von Emrich verwendet:

„Mult--Channel--Marketing ist die Nutzung mehrerer multifunktional vernetzter Kanäle sowohl für Kommunitkations als auch für Vertrieb von Produkten/Dienstleistungen eines Anbieters an organisationale Kunden bzw. Endverbraucher; es enthält mindestens zwei eigenständige unterschiedliche Kanäle für markierte Leistungsbündel mit einem Sortimentszusammenhang, für die ein kanalspezifischer Marketing--Mix bestehen kann und die in ein ganzheitliches Channel--Konzept integriert bzw. mit diesem kombiniert sind“\footnote{\cite{emrich}, S. 8}

\todo{ACHTUNG!!! Was steht da in Fußnote 15???}

\todo{
Multichannel-Marketing wird auch Multikanalstrategie genannt und bezeichnet die Kommunikations- und Vertreibsstrategie von Unternehmen, Personen aus der Zeilgruppe über unterschiedliche Kommunikations- und Vertriebsmaßnahmen zu erreichen.
\url{http://www.onlinemarketing-praxis.de/glossar/multichannel-marketing-multikanalstrategie}
}

\todo{
Der Kunde kann zwischen mehreren Kanälen wählen, z.B. stationärer Einzelhandel, Katalogversand, Onlineshop oder via TV, um Leistungen eines Anbieters nachzufragen. 
\url{http://wirtschaftslexikon.gabler.de/Archiv/119000/multi-channel-retailing-v5.html} 
}

\todo{
Multichannel-Marketing oder Multikanalstrategie ist der strategische Ansatz des Handels und der Dienstleister, die (potenziellen) Konsumenten auf mehreren verschiedenen Kommunikationskanälen zu erreichen und ist die konsequente Fortsetzung der Nutzung unterschiedlicher Werbekanäle nun in Form von Bereitstellung unterschiedlicher Kommunikations- und Vertriebswege.
\url{http://de.wikipedia.org/wiki/Multichannel-Marketing} 
}

\todo{
Mobile Marketing als Schlüsselgröße für Multichannel-Commerce
\url{http://link.springer.com/chapter/10.1007/978-3-322-90464-5_4}

\url{} \\
\url{} \\


}