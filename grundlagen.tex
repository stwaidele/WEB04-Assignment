\section{Grundlagen}
\label{sec:grundlagen}


\subsection{Definition Retailing}

\begin{quote}\textbf{retail:} „The sale of goods to the public in relatively small quantities for use or consumption rather than for resale.“\footnote{\cite{oxford:retail}}\end{quote}

Bei Manufactum handelt es sich nach obiger Definition um einen Akteur auf der Einzelhandelsebene: Die Waren werden in überschaubarer Menge an Endkunden verkauft. Die Vertriebswege Katalogversand, Ladengeschäft und Onlineshop sind ebenfalls im Einklang mit der von Wirz beschriebenen Einteilung\footnote{vgl. \cite{wirtz}, S.33}.

International spricht man bei der Kombination von Internetshop, Ladengeschäften und Katalogen von „Clicks, Bricks and Sheets“\footnote{vgl. \cite{klein}}.

\subsection{Definition Multi--Channel--Marketing}

In dieser Arbeit wird der Begriff Multichannel Marketing gemäß folgender Definitionen von Wirtz bzw. Emrich (siehe nächster Abschnitt) verwendet:

\begin{quote}„Unter \textbf{Multi--Channel--Marketing} versteht man den Prozess der Planung, DUrchführung und Kontrolle aller Marketingaktivitäten in einem Mehrkanalsystem. Dadurch sollen durch eine dauerhafte Befriedigung der Kundenbedürfnisse die Unternehmensziele verwirklicht werden.“\footnote{\cite{wirtz}, S21}\end{quote} 

\subsection{Definition Mehrkanalsystem}

Der Begriff „Mehrkanalsystem“ wird von Emrich in der oben zitierten Definition konkretisiert:

\begin{quote}„Mult--Channel--Marketing ist die Nutzung mehrerer multifunktional vernetzter Kanäle sowohl für Kommunitkations als auch für Vertrieb von Produkten/Dienstleistungen eines Anbieters an organisationale Kunden bzw. Endverbraucher; es enthält mindestens zwei eigenständige unterschiedliche Kanäle für markierte Leistungsbündel mit einem Sortimentszusammenhang, für die ein kanalspezifischer Marketing--Mix bestehen kann und die in ein ganzheitliches Channel--Konzept integriert bzw. mit diesem kombiniert sind.“\footnote{\cite{emrich}, S. 8}\end{quote} 

Aufgrund der Entwicklungen im Bereich Mobilfunk bzw. mobiles Internet wird dieser Kanal als besonders wichtig für Multichannel--Strategien angesehen\footnote{vgl. \cite{mohlenbruch}}.

\subsection{Verknüpfungsmöglichkeiten zwischen Kanälen}

Bei den drei Verkaufskanälen Webshop, Ladengeschäft und Katalogbesteht die Notwendigkeit die folgenden Verknüpfungen herzustellen:

\begin{itemize}
\item{\textbf{Katalog --- Ladengeschäft}\\
Der Hinweis auf persönliche Beratung in den Ladengeschäften ist bereits im Katalog vorhanden\footnote{siehe \cite{manufactum}, S. 403}. Ebenfalls sind die Öffnungszeiten, Adressen und Kontaktmöglichkeiten abgedruckt. Die einzelnen Standorte werden ausführlich und attraktiv vorgestellt.
}
\item{\textbf{Katalog --- Webshop}\\
Wie auf die Ladengeschäfte wird im Katalog ebenfalls auf die Angebote im Internet hingewiesen\footnote{siehe \cite{manufactum}, S. 396}. Auch der Internetshop wird vorgestellt. Die Verweise ins Internet werden durch die jeweilige \ac{URL} getätigt.\\
Eine weitere Möglichkeit der Übermittlung von URLs besteht in der Nutzung von QR--Codes, welche von Mobiltelefonen. Auch hiervon wird im Katalog Gebrauch gemacht, allerdings nur vereinzelt.
}
\item{\textbf{Ladengeschäft --- Katalog}\\
Durch die Auslage von Katalogen und Prospekten in den Ladengeschäften wird diese Verknüpfung hergestellt.
}
\item{\textbf{Ladengeschäft --- Webshop}\\
Kunden im Ladengeschäft können durch die Anbringung von \ac{URL}s oder QR--Codes auf Werbemitteln wie z.B. Preisschildern auf den Webshop aufmerksam gemacht werden.
}
\item{\textbf{Webshop --- Ladengeschäft}\\
Im Webshop sind außer den mit Print vergleichbaren Hinweisen noch spezielle Informationen bzgl. der Verfügbarkeit einzelner Produkte in den Geschäften denkbar. 
}
\item{\textbf{Webshop --- Katalog}\\
Im Webshop bzw. auf den Webseiten von Manufactum wird auf Bestell-- und Downloadmöglichkeiten der diveren Katalogen hingewiesen.
}
\end{itemize}