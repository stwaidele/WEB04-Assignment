\section{Fazit \& Ausblick}

\subsection{Fazit}

Durch Umsetzung der Empfehlungen in Kapitel \myref{hempf} ist es Manufactum möglich, die vorhandenen Vertriebskanäle optimal zu verbinden. Dem Kunden bietet sich somit ein flexibles Einkaufserlebnis, das von Öffnungszeiten, räumlicher Distanz aber auch technischer Ausstattung unabhängig ist. 

Da dem Kunden eine medienübergreifende Auswahl an Einkaufsmöglichkeiten geboten wird, können die beliebtesten Wege durch Beobachtung der eigenen Daten ermittelt und ausgebaut werden. Das Risiko, den Schritt vom Katalogversender zum digital ausgerichteten Fachgeschäft zu verpassen wird minimiert. Neue Kundengruppen können angesprochen werden.

Wie anhand der Bildbeispiele zu sehen ist, handelt es sich bei den beschriebenen Möglichkeiten nicht um exotisches Neuland, sondern um etablierte Techniken, die bereits vielfach im Einsatz sind. Somit bedeutet die Umsetzung wohl weniger die Schaffung von Wettbewerbsvorteilen, sondern eher um die Verhinderung von entsprechenden Nachteilen gegenüber der Konkurrenz.

\subsection{Ausblick}

Die folgenden Themengebiete wurden in der vorliegenden Arbeit nicht behandelt, stellen jedoch Aspekte dar, die im Zusammenhang von Multi--Channel--Marketing untersucht werden sollten:

Durch die Einbeziehung von digitalen Techniken ist es möglich, Kundenwünsche besser zu erkennen und darauf zu reagieren. Dies kann in Form von Targetting\footnote{siehe z.B. \cite{omptarget}} oder allgemeiner durch Analyse und Optimierung der Customer Journey\footnote{siehe z.B. \cite{ompcj}} geschehen. Auch entstehen immer wieder neue Möglichkeiten, digital mit Kunden in Kontakt zu treten\footnote{Etwa NFC, Bluetooth, iBeacon, aber auch Messengerdienste wie Whatsapp oder Facebook--Messenger. Vgl. \cite{waidele}}. Diese sollten regelmäßig erhoben und evaluiert werden, um hier stets die von der Zielgruppe genutzten Kanäle zu bedienen.


Des weiteren ist auch eine fortlaufende Untersuchung der Rentabilität der einzelnen Maßnahmen notwendig. Dies gilt sowohl für die bereits genutzten Vertriebsöglichkeiten als auch für die noch nicht Implementierten.
